\documentclass[margin,centered]{res}
\usepackage{pifont}
\usepackage{endnotes}
\let\footnote=\endnote

%\oddsidemargin -.5in
%\evensidemargin -.5in
\textwidth=5.0in
\itemsep=0in
\parsep=0in


\newenvironment{list1}{
  \begin{list}{\ding{71}}{%
      \setlength{\itemsep}{0in}
      \setlength{\parsep}{0in} \setlength{\parskip}{0in}
      \setlength{\topsep}{0in} \setlength{\partopsep}{0in}
      \setlength{\leftmargin}{0.17in}}}{\end{list}}
\newenvironment{list2}{
  \begin{list}{$\bullet$}{%
      \setlength{\itemsep}{0in}
      \setlength{\parsep}{0in} \setlength{\parskip}{0in}
      \setlength{\topsep}{0in} \setlength{\partopsep}{0in}
      \setlength{\leftmargin}{0.2in}}}{\end{list}}


\begin{document}



\name{Dimitris Aragiorgis}

\address{
\begin{tabular}{l}
\texttt{Email:} \textbf{dimitris.aragiorgis@gmail.com}\\
\texttt{Homepage:} http://cslab.ece.ntua.gr/$\sim$dimara/\\
\texttt{Telephone:} 00306944903954\\
\end{tabular}
}


\address{
\begin{tabular}{r}
\texttt{Date of Birth}: 21, August 1985\\
\texttt{Nationality}: Greek\\
\texttt{GPG:} 7A0811BD
% Arrikto Inc.\\
% 3505 El Camino Real, Palo Alto, CA 94306\\
% Mitropoleos 60, Athens, Greece\\
\end{tabular}
}



\begin{resume}

\section{\sc Education}
\begin{list1}
\item
{\bf Diploma in Electrical and Computer Engineering, ECE, NTUA},\\
Greece, Sep 2011
\item
{\bf Abitur/High School Diplom, Deutsche Schule Athen},\\
Greece, Sep 2003
\end{list1}

\section{Academia}
\rule[3pt]{\textwidth}{0.4pt}

\section{\sc Publications}
D. Aragiorgis, A. Nanos, and N. Koziris: \textbf{Coexisting Scheduling Policies boosting I/O Virtual Machines}, in Proceedings of the 6th Workshop on Virtualization in HighPerformance Cloud computing (VHPC 2011), held in conjunction with Euro-par 2011, Bordeaux, France.

\section{\sc Conference\\Attendee}
Cloud Computing Technology and Science 2011, Athens \footnote{CloudCom'11}\\
Virtualization for High-Performance Cloud Computing 2011, Bordeaux \footnote{VHPC'11}

\section{\sc Research Interests}
Computer Systems, Operating Systems, Virtualization, Scheduling, Profiling, Interconnects, Storage

\section{\sc Teaching\\Experience}
Teaching Assistance in Operating Systems, ECE, NTUA, 2011-2013

\section{Industy}
\rule[3pt]{\textwidth}{0.4pt}

\section{\sc Work\\Experience}
\begin{list1}
\item
  Staff Engineer, Arrikto 2021-present \footnote{http://www.arrikto.com}
\item
  System Administrator, Developer, Integration Manager, Arrikto, 2015-2021 \footnote{http://www.arrikto.com}
\item
  System Programmer of Synnefo Cloud IAAS software, GRNET, 2012-2015 \footnote{https://www.synnefo.org, https://okeanos.grnet.gr}
\item
  Design and Implementation of Visitor Management System, Air Force, 2013 \footnote{https://github.com/dimara/keda}
\item
  System Programmer for custom network applications, Prologic SA, 2008-2010
\item
  Customer Support for IT infrastructure, Prologic SA, 2006-2009
\end{list1}

\section{\sc Programming\\Experience}
\textit{Languages:} Python, C, bash \\
\textit{Frameworks:} Django, Kubernetes\\
\textit{Buildsystems:} CMake, Make, Webpack, Sphinx\\
\textit{Software Packaging:} Helm, Kustomize, Debian deb, Windows WiX\\
\textit{Configuration Management:} Ansible\\
\textit{Continuous Integration:} Buildbot, GitHub Actions \\
\textit{Debugging:} strace, tcpdump, gdb, pyrasite \\

\section{\sc Technical\\Skills}
\textit{Cloud Platforms:} AWS, GCP, Azure \\
\textit{Operating Systems:} Linux, MS Windows \\
\textit{Networking:} Istio, iptables, ferm, OpenVPN, PXE, openvswitch, vxlan\\
\textit{Services:} Apache2, nginx, Postfix, LDAP, dnsmasq\\
\textit{Virtualization:} QEMU, Xen, VMware ESXi, libvirt, OpenStack, Ganeti, Synnefo\\
\textit{Containers:} Docker, Kubernetes\\
\textit{Distributed Systems:} Ceph, etcd, Cassandra\\
\textit{Hardware Setup:} Physical Servers (HP, IBM, SuperMicro), Switches (HP, NetGear) \\

\section{\sc Basic\\Knowledge}
\textit{Web:} HTML, JavaScript, cgi-bin \\
\textit{DataBase Management Systems:} PostgreSQL, MySQL, CQL \\
\textit{Linux Kernel:} Scheduling, Network Devices, Block Devices, Device Mapper, procfs

\section{\sc Conference\\Attendee}
GanetiCon 2016, Dublin\\
GanetiCon 2015, Prague\\
GanetiCon 2014, Portland\\
Xen Hackathon 2013, Google Docks Dublin\\
GanetiCon 2013, Athens

\section{Public Presence}
\rule[3pt]{\textwidth}{0.4pt}

\section{\sc OpenSource\\Contributions}
\begin{list1}
\item
  KServe path-based serving \footnote{https://github.com/kserve/kserve}
\item
  Google Ganeti virtual machine cluster management tool \#4 contributor \footnote{https://github.com/ganeti/ganeti}
\item
  Synnefo IAAS cloud software \footnote{https://github.com/grnet/synnefo}
\item
  QEMU machine emulator and virtualizer \footnote{https://github.com/qemu/qemu}
\item
  NFQUEUE-based DHCP, DHCPv6 and RA server \footnote{https://github.com/grnet/snf-nfdhcpd}
\item
  Scapy python-based interactive packet manipulation library \footnote{https://github.com/secdev/scapy}
\item
  XenServer Windows Virtual Network Interface Device Driver \footnote{https://github.com/xenserver/win-xenvif}
\item
  OpenStack Volume discovery and local storage management lib \footnote{https://github.com/openstack/os-brick}
\end{list1}

\section{\sc Distinctions\\\& Awards}
\begin{list1}
\item
  Award for distinction in nation-wide contest on Physics 2003
\item
  Google's open source contributor award proposed by Iustin Pop 2014
\end{list1}

\section{Highlights}
\rule[3pt]{\textwidth}{0.4pt}

\section{Arrikto}

\section{\sc On-prem support}
Add support for deploying our software on-prem, including air-gapped
environments.

\section{\sc KServe performance tuning}
Do a performance analysis on KServe inference services, manage to find the
bottlenecks of vanilla setup and suggest tweaks for Istio and KNative to
improve request throughput and latency.

\section{\sc Cloud Integration}
Integrate and deploy our software on various Cloud Platforms (AWS, CGP, and
Azure) and also on premises (NVIDIA Bright).

\section{\sc MiniKF}
Design and implementation of a stripped down version of our software as a
pre-packaged VM on AWS, GCP, and Vagrant.

\section{\sc Software containerization}
Design and implementation of the containerization of our software
that used to be appliance-based.

\section{\sc Release management}
Design the build system of the software, how to version, package and distribute
it in the context of every commit is a release while emphasizing on a GitOps
process for deploying it.

\section{\sc Customer\\support}
Be on-call and deal with Sev1 customer issues in real time, debug systems at
scale, troubleshoot, and be able to identify even customer-side
misconfiguration on Cloud platforms or on-prem.

\section{\sc Full stack awareness}
Know the whole stack of the software, know how all components integrate with
each other, and as such be able to identify and troubleshoot pretty much any
issue.

\section{\sc Bug hunting}
Insist on finding the root cause of an issue, be able to reproduce it, and
finally identify where the bug is and what need to be done for solving it.


\section{GRNET}

\section{\sc Device Hotplug in Ganeti}
With Google Ganeti before v2.10, one had to reboot a VM in order to add/remove
devices NICs/Disks. Managed to submit and merge upstream a PR supporting
device hotplug for VMs running with the KVM hypervisor.

\section{\sc L2 isolated Network in Synnefo}
In synnefo, to be able to have private networks per user at first place we
had to rely on physical VLANs. But this had a limit of approximately 1000
private network. I designed a software implementation based on unique MAC
prefixes and ebtables rules.

\section{Air Force}

\section{\sc Booking management platform}
While doing military service, I designed and implemented a booking management
platform for a resort of National Air Force. They have been using it
since then, which translates to more that 10 years up-and-running.

\section{Misc}

\section{\sc BSOD on AWS}
Arrikto software supports instant backup and launching of VMs. Our customers
wanted to backup VMs running on on-prem, e.g., on ESXi, and then launch them to
AWS running Xen. Booting Linux VMs was pretty easy. Booting Windows resulted to
BSOD. And on AWS if it doesn't boot all you have is a screenshot. Nothing else.
In a nutshel, I had to simulate AWS EC2 environment locally, Xen Hypervisor,
proper PVHVM setup, patch iPXE software, inspect XenStore, use Windows kernel
debugger, patch official XenServer xenvif driver, install this drivers and
tweak windows registry offline using WinPE and finally watch it boot.

\section{\sc Packaging for Windows}
Managed to have CMake running on Linux generate a setup.exe for Windows using
PyInstaller, Wine and WiX toolset. This has been proved to be really helpful for
out CI/CD since on each commit we had the corresponding Windows installer.

\section{\sc Nested ESXi}
To be able to have ESXi servers on demand, primarily for development, I manage
to boot ESXi on QEMU. To do, and be able to take advantage of nested
virtualization features, I had to dig into QEMU and Linux kernel code, and
cherry-pick couple of commits to our production environment and only then we
could have Hardware CPU and MMU support for our nested VMs.


\section{Additional Info}
\rule[3pt]{\textwidth}{0.4pt}

\section{\sc Language\\Skills}
English (fluent), German (adequate), Greek (native)

\section{\sc Preferred\\Working\\Environment}
\textit{Operating System:} Linux (Debian)\\
\textit{Windows Manager:} xmonad\\
\textit{Versioning:} git\\
\textit{Writing:} Vim, Sphinx, \LaTeX\\
\textit{Mail:} Mutt

\section{\sc Military\\Obligations}
Fulfilled, Greek Air Force, 2013

\section{\sc Membership}
Greek Technical Chamber

% \section{\sc Driving License}
% Valid, Class B
%
\section{\sc Recommendation\\Letter}
Available upon request (Arrikto, GRNET, CSLab@NTUA, Greek Air Force)

\end{resume}

\def\enoteheading{\section{Notes}\rule[0pt]{\textwidth}{0.4pt}}
\parskip=2pt
\theendnotes

\end{document}
